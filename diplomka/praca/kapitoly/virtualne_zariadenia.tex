\chapter{Virtuálne zariadenia}
\label{chap:virt_zariadenia}

Dôležitú súčasť virtuálneho laboratória tvoria aj jeho zariadenia. Po analýze vyučovania sme mohli začať s ich získavaním a testovaním.




\section{Získavanie}

Zariadenia boli získavané z rôznych zdrojov. Predovšetkým boli použité zariadenia, ktoré sa už na katedre používali.

Ako už bolo naznačené v časti \ref{chap:adresarova_struktura}, EVE-ng podporuje rôzne typy zariadení. Patria medzi ne zariadenia typu Dynamips, Cisco IOL/IOU a QEMU.

Zoznam všetkých funkčných zariadení v nástroji EVE-ng je k dispozícii na CD médiu.




\subsubsection{Metodika}

Pri získavaní zariadení sme vychádzali zo zariadení, ktoré sa na katedre používajú a zo zoznamu zariadení, ktoré nástroj EVE-ng podporoval. Vybrané zariadenia následne prechádzali testovacími krokmi, ktoré sú bližšie popísané v ďalších častiach tejto kapitoly.

Všetky zozbierané zariadenia sú uložené v adresári \\ \texttt{/opt/unetlab/addons/rozne\_zariadenia}.




\section{Testovanie}
\label{chap:testovanie_zariadeni}

Testovanie zariadení bolo vykonané, aby sa zaistila kvalita a plynulosť vyučovacieho procesu, ako aj predvídateľnosť a replikovateľnosť behu jednotlivých zariadení a topológii.




\subsection{Testovanie spustiteľnosti}

Ako prvé bola testovaná spustiteľnosť zariadení. Na základe nej sa zistilo, či je zariadenie schopné zapnúť a konfigurovať ho pomocou vzdialeného prístupu.



\subsubsection{Metodika}

Najpr bolo potrebné pridať zariadenie na server. Môžeme na to použiť návody buď z oficiálnej EVE-ng stránky \cite{eve_ng_howtos}. Návody pre niektoré zariadenia sú k dispozícii aj na CD v adresári \emph{eve-ng}.

Vybrané zariadenia sme následne pridávali do predpripravenej topológie v EVE-ng. Následne sme vybrali jedno zariadenie, ktoré sme následne spustili. Ak sa zariadenie nespustilo, skúšali sme zistiť príčinu a vykonávali rôzne úpravy, akými by sme zariadenie v topológii spustili napr. modifikovať súbor, z ktorého sa zariadenie spúšťalo, opraviť oprávnenia, premenovať súbor do správneho formátu, upraviť systémové parametre na zariadenia a pod.

Ak sa zariadenie spustilo, skúsili sme sa pripojiť na jeho konzolu. Keď sa na ňu nedalo pripojiť, zariadenie sme zastavili a zmenili protokol na vzdialený prístup; obvykle stačilo vyskúšať protokoly \emph{telnet} a \emph{vnc}. V prípade, že zariadenie je v poriadku, mali by sme aspoň jedným z týchto protokolov dostať textový resp. grafický výstup konzoly na zariadení.

Ak sme sa nakoniec pripojili ku konzole zariadenia, sledovali sme jeho spúšťanie. Čakali sme na dokončenie spúšťania. V prípade, že sa zariadenie úspešne spustilo, prihlásili sme da doň predvolenými prihlasovacími údajmi, ak to bolo potrebné, a vyskúšali sme, či konzola reaguje na vstup z klávesnice. Ak konzola reagovala, zariadenie zostalo uložené na serveri. Ak sme ku zariadeniu nevedeli zistiť prihlasovacie údaje, umiestnili sme ho do osobitného adresára.

Ak sa ani po týchto úkonoch zariadenie nespustilo, odstránili sme ho zo servera. Týmto spôsobom sme získali množinu spustiteľných zariadení v EVE-ng topológii.

\subsubsection{Vyhodnotenie}

Výsledky testovania spustiteľnosti zariadení sú zhrnuté v súbore \\ \emph{sumarny\_prehlad\_podporovanych\_zariadeni\_vo\_virtualnych\_sietovych\_nastrojoch.ods}. Súbor obsahuje viacero stĺpcov, ktorých význam je bližšie vysvetlený v tabuľke \ref{tab:sumarny_prehlad_stlpce}. Z výstupov tohto testovania bol vytvorený aj skript na úpravu šablón, ktorý je bližšie opísaný v závere tejto časti.

\begin{longtabu} to \textwidth {| X[2.5,l,m] | X[5.0,l,m] |}
\caption{Stĺpce v sumárnom prehľade zariadení}
\label{tab:sumarny_prehlad_stlpce} \\
\hline
    \multicolumn{1}{|c|}{\textbf{Stĺpec}} & \multicolumn{1}{|c|}{\textbf{Popis}} \\
\hline
    \textbf{Zariadenie} & \makecell[lc]{Názov alebo modelové označenie zariadenia.} \\
\hline
    \textbf{Platforma} & \makecell[lc]{Názov a verzia operačného systému zariadenia.} \\
\hline
    \textbf{Názov súboru} & \makecell[lc]{Pomenovanie zariadenia na serveri.} \\
\hline
    \textbf{Spôsob virtualizácie} & Hypervízor, pod ktorým zariadenie môže byť spustené. \\
\hline
    \textbf{\makecell[lc]{Predvolené prihlasovacie \\ údaje}} & Predvolené prihlasovacie meno a heslo, ak to zariadenie vyžaduje. \\
\hline
    \textbf{Spôsob pripojenia} & \makecell[lc]{Protokol, ktorý sa používa na vzdialenú konfiguráciu \\ zariadenia.} \\
\hline
    \textbf{Úspešné spustenie} & Informácia, či sa zariadenie v topológii spustilo. \\
\hline
    \textbf{Poznámky} & \makecell[lc]{Bližší popis správania sa daného zariadenia, popr. \\ problémy pri používaní zariadenia a ich možné riešenie. \\ Ďalej sa v ňom môžu nachádzať niektoré podporované \\ technológie a internetové odkazy ako zdroj pre informácie \\ uvádzané pre dané zariadenie.} \\
\hline
\end{longtabu}

Ďalšie stĺpce slúžia pre prieskum spustiteľnosti zariadení v rôznych nástrojoch nasadených na rôznych platformách.

Pokiaľ sme zistili, že protokol vzdialeného prístupu sa odlišuje od predvoleného protokolu v šablóne na EVE-ng serveri, túto šablónu sme museli upraviť ručne. Avšak s postupným nárastom testovaných zariadení rástol aj počet úprav v šablónach pre jednotlivé zariadenia. Preto sme sa rozhodli vytvoriť skript, ktorý by celý proces automatizoval. Tento skript na úpravu šablón upravuje jednotlivé atribúty konkrétnym zariadeniam. Výsledky testovania sa prejavili v skripte na úpravu šablón, v ktorom  boli nastavené atribúty pre protokol na vzdialený prístup k zariadeniam a počte a type rozširujúcich kariet pre zariadenia.




\subsection{Testovanie systémových požiadaviek}
\label{chap:testovanie_zariadeni_benchmark}

Testovanie systémových požiadaviek zariadení bolo realizované na fyzickom EVE-ng serveri. Tento druh testovania bol dôležitý preto, aby sme mohli danému zariadeniu nastaviť dostatočné technické parametre, čím zaistíme jeho plynulý chod a predídeme rôznym komplikáciám počas používania vo vyučovaní. Tieto parametre sú uložené v šablóne pre dané zariadenie.

Úprava šablón je realizovaná skriptom, po ktorého vykonaní sa príslušným zariadeniam zmenia konkrétne technické parametre v šablóne. Po úprave šablón sa zmeny prejavia okamžite a po pridaní zariadenia do topológie, kedy sa jeho parametre automaticky nastavia na správne hodnoty. Pre vybrané zariadenia boli merané tieto veličiny:

\begin{itemize}[noitemsep]
    \item Vyťaženie procesora
    \item Využitie operačnej pamäte
    \item Vyťaženie pevného disku
\end{itemize}

Vyťaženie procesora bolo merané z dôvodu jeho intenzívnej činnosti hlavne počas spúšťania zariadení, ale môže byť vyťažovaný aj po dokončení spúšťania. Na základe toho budeme vedieť určiť, koľko zariadení budeme môcť v topológii spustiť naraz, a koľko už spustených zariadení zvládne server spravovať celkovo. Pri meraní vyťaženia procesora bolo merané celkové vyťaženie aj vyťaženie jednotlivých jadier procesora.

Operačná pamäť je najviac využitá po dokončení spúšťania zariadenia. Kapacita operačnej pamäte ovplyvňuje celkový počet spustených zariadení na serveri. Meranie jej vyťaženia je pomerne jednoduché pre celý systém, ale pre tieto účely sme potrebovali s vysokou presnosťou vedieť, do akej koľko operačnej pamäte využíva iba konkrétne zariadenie.

Na meranie využitia operačnej pamäte boli použité dva nástroje: \texttt{ps} a \texttt{ps\_mem}. Prvý z nich už bol na EVE-ng serveri prítomný, druhý bolo potrebné nainštalovať dodatočne. Na meranie boli použité dva nástroje, aby sa navzájom výsledky oboch nástrojov medzi sebou validovali.

Disk najviac vyťažený predovšetkým pri spúšťaní zariadenia, ale môže byť vyťažený aj po dokončení spúšťania. Vyťaženie disku bolo do merania zahrnuté, aby sme vedeli odhadnúť, do akej miery je spúšťanie a beh zariadení ovplyvnený rýchlosťou pevného disku a o koľko by sa teoreticky zvýšil výkon servera pri ich výmene za SSD disky.

Z merania boli vynechané meranie frekvencie procesora a vyťaženie sieťového rozhrania. Frekvencia procesora bola vynechaná, lebo procesor podľa príkazu \texttt{watch lscpu | grep "MHz"} striedal iba dve frekvencie: 2000.000 MHz (minimálna frekvencia) a 2333.000 MHz (maximálna frekvencia).

Vyťaženie sieťovej karty je zanedbateľné pri meraní výkonnosti jednotlivých zariadení, keďže sa sieť využíva iba na interakciu používateľa s klientskou aplikáciou, čo vytvára zanedbateľnú záťaž.

Na monitorovanie vybraných veličín sú určené rôzne nástroje a stratégie. Mohli sme napr. vytvoriť skript, ktorý by pomocou viacerých špecializovaných nástrojov meral vyťaženie jednotlivých prvkov systému, alebo použiť nástroj, ktorý je schopný monitorovať široké spektrum systémových zdrojov. Zvažovali sme tieto nástroje:

\begin{itemize}[noitemsep]
    \item \textbf{iotop} ~~~~~ - ~~monitorovanie procesov podľa využitia disku
    \item \textbf{nmap} ~~~~ - ~~monitorovanie sieťovej prevádzky
    \item \textbf{nethogs} ~ - ~~monitorovanie procesov podľa využitia sieťového rozhrania
    \item \textbf{dstat} ~~~~~~ - ~~monitorovanie rôznych systémových prostriedkov
    \item \textbf{sysstat} ~~~~- ~~monitorovanie rôznych systémových prostriedkov
    \item \textbf{netdata} ~~~- ~~monitorovanie rôznych systémových prostriedkov cez web rozhranie
\end{itemize}

Z uvedených nástrojov sme sa rozhodli použiť nástroj \emph{dstat}. Hlavným dôvodom, prečo sme si sme sa rozhodli ďalej používať a upravovať tento nástroj a uprednostniť ho pred ostatnými nástrojmi, bola možnosť ovládania cez príkazový riadok a ukladanie ziskaných údajov do CSV súboru. Dáta sa do CSV súboru zapisovali každú sekundu. Následne sme si vytvorili tabuľkový dokument, ktorý vyhodnocoval namerané dáta z CSV súboru. Vyhodnocovaním nameraných údajov sa budeme bližšie zaoberať v časti \ref{chap:testovanie_zariadeni_benchmark_vyhodnotenie}.

Nástroj \emph{dstat} však bolo potrebné upraviť tak, aby zisťoval využitie operačnej pamäte iba pre zariadenia v EVE-ng topológii, nie celého systému.

Pre tento účel sme vytvorili kópiu nástroja \emph{dstat} s názvom \emph{dstat\_custom}. V tejto kópii sme následne upravovali jeho zdrojový kód pre naše potreby. Nástroj je vytvorený v programovacom jazyku Python.

Nástroj \emph{dstat\_custom} bol upravený tak, že sa meria využitie operačnej pamäte pre zariadenie v EVE-ng topológii nástrojmi \emph{ps} a \emph{ps\_mem}. Obidva príkazy merajú tú istú množinu procesov patriacich spusteným zariadeniam v EVE-ng topológii, aby sme mohli merania jednotlivých nástrojov medzi sebou porovnať. Hodnoty namerané obomi nástrojmi by mali byť približne rovnaké na to, aby sme mohli výsledky považovať za validné.

Vo výslednom CSV súbore vytvorenom nástrojom \emph{dstat\_custom} sa stĺpce \emph{used} a \emph{buffered} v časti \emph{memory usage} nahradia stĺpcami \emph{MemUsed-ps} a \emph{MemUsed-ps\_mem}.

Aby nástroj \emph{dstat\_custom} fungoval správne, musia byť nainštalované balíčky \emph{dstat}, \emph{ps\_mem} a \emph{sultan}. Význam prvých dvoch balíčkov už bol v tejto časti vysvetlený. Balíček \emph{sultan} slúžil na vykonávanie terminálových príkazov zvnútra Python programu.

Už upravený nástroj \emph{dstat\_custom} je možné nájsť na zálohovacom serveri v adresári \emph{eve\_ng\_specific/usr\_bin/dstat\_custom}.





\subsubsection{Metodika}

Najprv bol vytvorený zoznam vybraných zariadení, ktorých systémové požiadavky sme sa rozhodli testovať. Ten sa nachádza na priloženom CD v súbore \\ \emph{eve\_ng/13\_1\_profiling\_and\_benchmarking\_device\_list.txt}. Zoznam zariadení bol vytvorený na základe kapitoly \ref{chap:analyza_vyucovania} - \nameref{chap:analyza_vyucovania}, keďže malo zmysel testovať iba zariadenia, ktoré by mohla katedra použiť vo vyučovaní.

Meranie systémových zdrojov zariadenia bolo vykonané nástrojom \emph{dstat\_custom}. Pred začiatkom každého merania bolo potrebné vykonať kroky, ktoré zvyšujú konzistenciu a presnosť výsledkov a reprezentujú najhoršie možné podmienky pre beh zariadenia. Všetky zariadenia sú uložené na rovnakom pevnom disku ako operačný systém.

Pred začatím merania výkonnosti každého zariadenia bolo vypnuté používanie \emph{swap} partície a vyprazdnené vyrovnávacie časti operačnej pamäte (\emph{cache} a \emph{buffer}).

Na EVE-ng serveri bola počas celého merania vypnutá funkcia UKMS (Universal Kernel Samepage Merging). UKSM je mechanizmus, ktorý umožňuje šetriť využitie operačnej pamäte, keď je spustených viacero zariadení rovnakého typu. Ak je Ale UKMS aktívne a spustíme viac ako 10 QEMU zariadení, ich výkon by mohol byť v dôsledku tohto mechanizmu znížený \cite{eve_ng_faq}.

\noindent
Meranie systémových požiadaviek zariadenia prebiehalo nasledovne:

\begin{enumerate}[noitemsep]
    \item Do EVE-ng topológie sme pridali jedno zariadenie.
    \item \label{nastavenie_sys_param} Nastavíme mu maximálne množstvo operačnej pamäte, s ktorým je zariadenie schopné, úspešne sa spustiť. Ak je to možné, nastavíme počet procesorov na 1 CPU.
    \item Vykonáme úvodné kroky pred meraním.
    \item Nástrojom \emph{dstat\_custom} spustíme sledovanie systémových prostriedkov, ktorý bude ukladať namerané údaje do súboru.
    \item Spustíme zariadenie.
    \item Z nástroja zistíme čas spustenia zariadenia a zapíšeme si ho do osobitného súboru napr. \emph{boot\_time.txt}.
    \item Pripojíme sa na konzolu zariadenia. Počkáme, kým neuvidíme interaktívny príkazový riadok (CLI) alebo výzvu na prihlásenie.
    \item Ak je to nutné, po úspešnom spustení zariadenia sa naň prihlásime.
    \item Akonáhle uvidíme interaktívny príkazový riadok (CLI), do súboru \emph{boot\_time.txt} si uložíme čas, v ktorom zariadenie dokončilo svoje spúšťanie. 
    \item Zariadenie necháme 1-3 minúty stabilizovať.
    \item Ukončíme sledovanie systémových prostriedkov.
    \item Zastavíme zariadenie.
\end{enumerate}

Pokiaľ sa vyskytli komplikácie so spúšťaním alebo behom zariadenia, vrátili sme sa na krok \ref{nastavenie_sys_param} a stanovili sme pre zariadenie iné systémové parametre. Systémové parametre zariadenia sme menili dovtedy, kým sa zariadenie nespustilo a odozva z klávesnice z jeho konzoly bola prijateľná. Podrobnejší popis rôznych konfigurácii systémových parametrov zariadenia je k dispozícii na CD v súbore \emph{eve\_ng/13\_1\_profiling\_and\_benchmarking\_device\_list.txt}.

Po skončení merania vznikli dva nové súbory: súbor s nameranými údajmi a súbor s trvaním spúšťania zariadenia. Tieto súbory tvorili vstup pre tabuľkový súbor na vyhodnocovanie nameraných údajov pre zariadenie, ktorému sa venujeme v nasledujúcej časti.

Po ukončení merania všetkých zariadení môžeme znova zapnúť "swap" partíciu.




\subsubsection{Vyhodnotenie}
\label{chap:testovanie_zariadeni_benchmark_vyhodnotenie}

Po vytvorení súboru s nameranými údajmi a súboru s trvaním spúšťania zariadenia môžeme vložiť ich údaje do tabuľkového dokumentu. Na základe vstupných údajov sa automaticky prepočítajú výstupné hodnoty vo všetkých tabuľkách tohto dokumentu. Nižšie je opísaný priebeh vyhodnotenia nameraných údajov:

\begin{enumerate}[noitemsep]
    \item Do hárku \emph{SuroveUdaje} vložíme dáta namerané nástrojom \emph{dstat\_custom}.
    \item Do hárku \emph{VstupVystup} vložíme tieto údaje
    \begin{itemize}[noitemsep]
        \item Do poľa \emph{Start} zadáme čas, kedy sme zariadenie spustili.
        \item Do poľa \emph{Stop} zadáme čas, kedy zariadenie dokončilo svoje spúšťanie a zobrazilo interaktívny príkazový riadok (CLI).
        \item Do poľa \emph{Množstvo voľnej RAM na serveri (MB)} zadáme množstvo voľnej operačnej pamäte na EVE-ng serveri v stave pokoja.
    \end{itemize}
\end{enumerate}

Po zadaní spomenutých vstupných údajov tabuľkový dokument poskytne v hárku \emph{VstupVystup} odpovede na tieto otázky:

\begin{itemize}[noitemsep]
    \item Čas spúšťania - čas potrebný na dokončenie spúšťania zariadenia
    \item Odhadované množstvo operačnej pamäte pre jedno zariadenie / topológiu (MB)
    \item Odhadovaný počet zariadení, ktoré je možné naraz spustiť na základe celkového vyťaženia CPU
    \item Odhadovaný počet sputených zariadení na základe celkového vyťaženia CPU
    \item Odhadovaný počet sputených zariadení na základe voľnej RAM
\end{itemize}

Od všetkých vstupných údajov boli pred ďalším spracovaním odčítané namerané hodnoty z pokojového stavu EVE-ng servera.

Výsledky merania systémových požiadaviek zariadení reprezentujú najhorší možný scenár behu zariadenia.

Výsledky testovania vybraných zariadení sú prítomné na CD v adresári \emph{eve-ng/profiling\_and\_benchmarking\_results}.

Podľa tabuľkového dokumentu bolo pre každé zariadenie nastavené množstvo operačnej pamäte a počet CPU v skripte na úpravu šablón. Zariadenie vďaka tomu môžeme pridávať do topológie bez toho, aby sme museli premýšľať, či má nastavené dostatočné systémové parametre.




\subsection{Testovanie technológii}
\label{chap:testovanie_technologii}

V tejto časti využijeme poznatky z kapitoly \ref{chap:analyza_vyucovania} a použijeme ich na testovanie, či vybrané zariadenia podporujú témy vyučované na katedre. Boli testované podporované technológie iba na Cisco zariadeniach, keďže tie sa používajú vo vyučovaní najviac.

Cieľom testovania technológii je zistiť, do akej je možné použiť virtuálne zariadenia pri vyučovaní takých tém, na ktoré boli doposiaľ použité fyzické zariadenia alebo virtualizačné riešenia s užším rozsahom podporovaných technológii. Jednalo sa najmä o podporu prepínacích technológii na predmetoch \emph{Pokročilé prepínanie v informačno-komunikačných sieťach} (CCNP) a \emph{Počítačové siete 1} (CCNA). Prioritne boli testované témy vyučované na CCNP kurze, keďže prepínacie technológie v CCNP kurze zahŕňajú aj tie zo CCNA.



\subsubsection{Metodika}

-zdokumentovat zistovanie features pre zariadenia
        -ako som postupoval, zdroje, co som porovnaval, ako som vyberal zariadenia, na ktore oblasti katedry som sa zameriaval (pozret sa v telefone na obrazok tabule od veduceho: predmety a ich problematicke casti + zariadenia na ne)

Na testovanie podporovaných technológii vybraných zariadení bol vytvorený skript. Skript sa skladal z konfiguračných príkazov, ktoré mali za úlohu testovať ich podporu na zariadení. Časti tohto skriptu boli postupne zadávané do konfigurácie zariadenia. Skript je prítomný na CD v súbore \\
\emph{materialy\_na\_predmety/0\_2\_vyucovane\_technologie\_testovaci\_skript\_cisco.txt}.



\subsubsection{Vyhodnotenie}

Po vykonaní konfiguračných príkazov zo skriptu bolo potrebné vyhodnotiť, či a do akej miery je testovaná technológia podporovaná. Technológia bola označená ako podporovaná vtedy, ak príkaz nevypísal žiadne chybové hlásenie. V opačnom prípade sa testovali alternatívne konfigurácie. Ak ani žiadna z alternatívnych konfigurácii nebola podporovaná, technológia bola vyhodnotená ako nepodporovaná. Výsledky testovania technológii sú dostupné na CD v súbore \\
\emph{materialy\_na\_predmety/0\_0\_vyucovane\_technologie.ods}. Podporované vyučované témy testovaných zariadení sú farebne odlíšené, popr. doplnené krátkym komentárom.

Nižšie je uvedený zoznam vybraných predmetov a k nim prislúchajúce odporúčané zariadenia.

\begin{itemize}[noitemsep]
    \item Počítačové siete 1
    \begin{itemize}[noitemsep]
        \item Cisco IOL prepínač
        \item Cisco vIOS prepínač
    \end{itemize}
    \item Počítačové siete 2
    \begin{itemize}[noitemsep]
        \item Cisco IOL smerovač
    \end{itemize}
    \item Projektovanie sietí 1
    \begin{itemize}[noitemsep]
        \item Cisco IOL smerovač
        \item Cisco 3725
        \item Cisco 7200
        \item Cisco vIOS smerovač
        \item Cisco CSR
        \item Cisco prepínač (IOL switch)
        \item Cisco vIOS prepínač (QEMU)
    \end{itemize}
    \item Projektovanie sietí 2
    \begin{itemize}[noitemsep]
        \item Cisco IOL smerovač
        \item Cisco CSR a CSR-ng
        \item Nokia VSR
    \end{itemize}
    \item Pokročilé prepínanie v informačno-komunikačných sieťach
    \begin{itemize}[noitemsep]
        \item Rovnaké zariadenia, ako pri predmete Počítačové siete 1.
    \end{itemize}
    \item Pokročilé smerovanie v informačno-komunikačných sieťach
    \begin{itemize}[noitemsep]
        \item Rovnaké zariadenia, ako pri predmete Projektovanie sietí 1.
    \end{itemize}
\end{itemize}

Cisco IOL smerovač ako jediný v EVE-ng obsahuje sériové rozhrania, preto ako jediný v EVE-ng podporuje Point-to-point technológie. Pre výučbu HSRP a VRRP protokolov sú odporúčané Cisco IOL a Cisco vIOS prepínač. Aplikačné protokoly, ako napr. NTP, DHCP alebo Syslog, sú, až na malé výnimky, podporované na všetkých Cisco prepínačoch a smerovačoch. GLBP IPv4 je podporované na všetkých testovaných zariadeniach, ale GLBP IPv6 podporujú iba smerovače Cisco 7200, Cisco IOL smerovač, Cisco vIOS smerovač a Cisco CSR-ng. Cisco podporuje VPLS iba na smerovačoch Cisco CSR a CSR-ng. IP SLA plne podporuje iba smerovač Cisco 3725.

Zoznam odporúčaných zariadení slúži iba na odhad, ktoré zariadenia majú najväčšiu pravdepodobnosť využitia na danom predmete. V prípade, že má predmet v zozname viacero zariadení, treba sa na základe ďalších kritérii rozhodnúť, ktoré z nich budú použité v topológii. Môžeme napr. brať do úvahy iné vyučované témy v topológii, systémové požiadavky zariadenia, iné technické obmedzenia zariadenia a pod. Tento zoznam bol použitý v poslednej fáze projektu - \nameref{chap:nasadenie_do_vyucovania}, ktorej sa venujem v kapitole \ref{chap:nasadenie_do_vyucovania}.