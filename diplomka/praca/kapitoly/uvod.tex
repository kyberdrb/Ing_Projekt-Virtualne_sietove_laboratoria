\chapter*{Úvod}
\addcontentsline{toc}{chapter}{Úvod}

Virtualizácia sa stáva vo svete čoraz populárnejšou. Využíva sa v rôznych oblastiach napríklad v tzv. Cloud Computing alebo Software Defined Networking. Virtualizáciou sa zaberá aj Katedra informačných sietí Žilinskej univerzity v Žiline. Jedným z projektov, kde sa virtualizácia využíva, je projekt virtuálneho sieťového laboratória. 

Našou úlohou je preskúmať existujúce riešenia v oblasti virtuálnych sieťových laboratórii a porovnať ich podľa vopred stanovených kritérií. Na základe toho z nich vyberieme jedno riešenie, ktorým sa budeme v práci ďalej zaoberať. Pre vybrané sieťové laboratórium následne vypracujeme návody na inštaláciu, úpravu, používanie a nasadenie do infraštruktúry katedry. Následne analyzujeme kompatibilitu nástroja s rôznymi zariadeniami. Vybrané zariadenia otestujeme, do akej miery vyťažujú systémove zdroje, aby im moholi byť nastavené primerané parametre pre spoľahlivý chod. Následne si vyberieme predmety, na ktoré má byť tento nástroj použítý a analyzujeme technológie, ktoré sa na nich vyučujú. Nakoniec zistíme, ktoré zariadenia sú pre daný predmet vhodné.

Virtuálne sieťové laboratórium bude nástrojom, ktorý zefektívni a skvalitní výučbu sieťových technológii vo vybraných predmetoch na Katedre informačných sietí.

Tému diplomovej práce som si vybral predovšetkým preto, aby som pomohol dosiahnuť skvalitnenie a zefektívnenie vyučovacieho procesu na katedre. Tak budú mať učitelia aj študenti jednotnú platformu pre vyučovanie, ktorá učiteľom umožní jednoducho vytvárať modelové situácie a študentom uľahčí ich pochopenie.