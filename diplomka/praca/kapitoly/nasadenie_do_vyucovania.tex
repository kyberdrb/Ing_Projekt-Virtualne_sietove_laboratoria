\chapter{Nasadenie do vyučovania}
\label{chap:nasadenie_do_vyucovania}

Aby bola preverená celková kvalita virtuálneho laboratória, bol nástroj EVE-ng nasadený na vypracovávanie rôznych topológii z vybraných predmetov. 




\section{Používanie EVE-ng}
\label{chap:pouzivanie_eve_ng}

EVE-ng poskytuje na správu topológii a používateľov webové rozhranie. Webové rozhranie je dostupné v 2 módoch: natívnom a HTML5. HTML5 mód zabezpečuje integráciu pomocou reverzného proxy servera \emph{Apache Guacamole}, ktorý sa pripája na konzoly zariadení. HTML5 mód web rozhrania EVE-ng bol však menej stabilný a reagoval výrazne pomalšie pri práci s topológiou v provonaní s natívnym módom. Preto bolo webové rozhranie ďalej používané iba v natívnom móde.

V HTML5 móde sa na obrazovke s otvorenou topológiou po kliknutí na zariadenie otvorí jeho vzdialená konzola. V natívnom móde potrebujeme mať pre otvorenie konzoly na zariadení nainštalovaný tzv. integračný balíček. Predvolená vs KIS verzia integračného balíčka -> SSH tunely (pre vzdialený prístup k zariadeniam v topológii).



-popisat
      -vytvorenie topologie
        -z GUI (pridavanie zariadenia)
        -editovaním UNL súboru s topológiou
            -popisat hlavne parametre, ktore sa lisia pri duplikovani textovej reprezentacie danej topologie
        -vyslovit odporucanie na vytvaranie topologii
            -> GUI -> Nodes (nazvy zariadeni) -> UNL (skraslenie topologie upravenim suradnic pre zariadenia) - VID A4 PAPIER "DUPLIKÁCIA TOPOÓGIE V EVE-NG -> ZÁVER"
      -pridelovanie portovych cisel zariadeniam
        -rozsah
        -pridelovanie portovych cisel je sekvencne
      -spustanie zariadeni
        -po jednom
        -vybrana skupina
          -Ctrl+klik -> pravy klik -> Start Selected
          -oznacenie mysou -> pravy klik -> Start selected
        -vsetky zariadenia v topologii
          -More actions -> Start all nodes
      -odhadnut systemove poziadavky pre topologie z vybranych predmetov a na zaklade toho odhadnut minimalne systemove poziadavky servera
      -> začať topológiami z predmetov, na ktorých bol nástroj nasadený.




\section{Počítačové siete 1 a 2}

V rámci predmetu Počítačové siete 2 bol nástroj EVE-ng nasadený do vyučovania na vypracovávanie topológii s \emph{point-to-point} technológiami. Topológie boli spustené na fyzickom EVE-ng serveri. V prípade zlyhania EVE-ng boli pripravené aj záložné topológie v overenom nástroji Dynamips/Dynagen.

Najprv bola vytvorená základná topológia. Tá pozostávala zo štyroch Cisco IOL smerovačov a dvoch koncových zariadení s operačným systémom Alpine Linux. Cisco IOL smerovač bol vybraný, pretože ako jediný podporoval sériové rozhrania a \emph{point-to-point} technológie. Koncové zariadenie Alpine Linux bolo vybrané pre svoju nenáročnosť na systémové zdroje.

{\huge TODO- obrázok základnej P2P topo}

Celkovo bolo vytvorených 8 zhodných topológii, ktoré medzi sebou zdieľali jeden učiteľský smerovač. V topológii sa celkovo nachádzalo 33 Cisco IOL smerovačov a 16 koncových staníc.

{\huge TODO- obrázok celkovej P2P topo}

IOL smerovače fungovali, až na príkaz \texttt{clock rate} na sériových rozhraniach, bez chyby.


Ukázať, že to závisí od párnosti čísla skupiny - párne DTE, nepárne DCE, nie od poradia skupín so sériovými rozhraniami t.j. párne číslo skupiny sériových rozhraní bude vždy DTE, nepárne vždy DCE. Nanešťastie nastavenie DTE/DCE módu pre sériové rozhrania nie je v EVE-ng pri Cisco IOL smerovačoch automatické. {\huge Je automatické v Dynamips/Dynagene?}

{\huge TODO- obrázky s DCE/DTE seriákmi na IOL routri - 2E8S, 1E8S, 0E8S}

V jednej skupine sa vyskytol problém s jednosmernou PAP autentifikáciou študentského smerovača voči učiteľskému (R\_Ucitel(s4/1)-5R2(s2/1)). Príkaz \texttt{debug ppp authentication} hlásil chybu pri autentifikácii. Riešenie spočívalo v odstránení používateľa, vypnutí \emph{ppp} konfigurácie a vypnutí rozhraní. Tieto kroky boli vykonané aj na učiteľskom, aj na študentskom smerovači. Následne sa konektivita obnovila a spojenie pomocou PAP autentifikácie sa úspešne nadviazalo.

Je možné, že problémy vznikli aj kvôli tomu, že medzi študentským a učiteľským smerovačom boli na oboch stranách sériové rozhrania párnej skupiny t.j. obidva konce linky boli typu DTE. Niektoré skupiny študentov boli tiež pripojené k učiteľskému smerovaču sériovým rozhraním z párnej skupiny, ale takéto problémy nezaznamenali. Podobne tomu bolo aj pri prepojení Cisco IOL smerovačov rozhraniami DCE.

Z toho vyplýva, že Cisco IOL smerovače v EVE-ng majú pri prepojení dvoch smerovačov sériovou linkou s rovnakým módom nedefinované správanie. Tomu sa dá predísť vhodným návrhom topológie. Ten spočíva v tom, že sériové rozhrania medzi smerovačmi kombinujeme tak, aby bolo prepojené vždy sériové rozhranie párnej skupiny na jednom smerovači so sériovým rozhraním nepárnej skupiny na inom smerovači t.j. \emph{Serial2/x} (DCE) na prvom smerovači sa musí pripojiť napr. k \emph{Serial3/x} na druhom.

Pridať aj troubleshooting a vypisy z routrov (ale iba velmi kratke), aby bolo jasne, v com bol problem a ako sme ho riesili (vid otvorene konzoly routrov na ploche 1)
  
  

-vyhodnotit dstat csv - (8 topologii = 32 Cisco IOL smerovačov (4*8 + 1 učiteľský) + 16 QEMU zariadení (Alpine Linux x86))
        -> VYHODNOTIT CEZ LIBRE CALC
        
    
    
{\huge TODO - TOTO DAŤ DO KAPITOLY ÚPRAVA NÁSTROJA EVE-NG}
-zistený limit - maximálny počet zariadení obmedzený na 63; 64. zariadenie sa chvíľu tvári, že sa spúšťa, ale po niekoľkých sekundách sa automaticky vypne.

{\huge TODO - AJ TOTO DAŤ DO KAPITOLY ÚPRAVA NÁSTROJA EVE-NG}
-EVE-ng v momentalnom stave vie zatvorit topologiu, ale nevie po zatvoreni jednej topologie (co teraz funguje), pracovat so zariadeniami v inej topologii. Z toho vyplýva, že v EVE-ng môžeme mať spustenú iba jednu topológiu, ktorá je navyše obmedzená na 63 zariadení, 64. sa síce začne spúšťať, ale hneď na to sa vypne.
    ->DOCASNE RIESENIE spociva v rozsireni portoveho rozsahu pre jednu topológiu. To vyrieši problém s obmedzením počtu zariadení v jednej topológii, ale nerieši to problém s nezávislým spúšťaním topológii ako jeden používateľ.



\section{Projektovanie sietí 1}

náhrada/doplnok Dynamips servera (s takym vykonom, t.j. cca 10/15skupin po 10 routroch)


Maximálny počet spustených topológii daného typu odhadneme sčítaním ich systémových požiadaviek. Mali by z toho vyplynúť odhad minimálnych systémových požiadaviek servera pri topológiách na vybrané predmety.



\section{Projektovanie sietí 2}

VMware inštalácia

V EVE-ng boli úspešne dokončené semestrálne práce s využitím technológii VPLS a Seamless MPLS.

{\huge TODO- replikovať jakubovu topológiu v EVE-ng - konfiguráky => FB, topológia => gmail konverzácia}




\section{CCNP Routing}





\section{CCNP Switching}




\section{Vyhodnotenie}

Ukázalo sa, že ...