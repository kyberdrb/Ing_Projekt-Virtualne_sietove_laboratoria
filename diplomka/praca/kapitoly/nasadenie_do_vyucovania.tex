\chapter{Nasadenie do vyučovania}
\label{chap:nasadenie_do_vyucovania}

Rozpísať, ako sa ktorý nástroj správal pri vypracovávaní úloh z daného predmetu.

Urobiť napríklad scenár: syslog a snmp server z inych hostov a routrov

Maximálny počet spustených topológii daného typu odhadneme sčítaním ich systémových požiadaviek. Mali by z toho vyplynúť odhad minimálnych systémových požiadaviek servera pri topológiách na vybrané predmety.

\section{Integračný balíček EVE-ng}
\label{chap:nasadenie_klient}

Web rozhranie - 2 módy - Natívny a HTML5.
Integračný balíček je nutný pre Natívny režim. HTML5 mód zabezpečuje integráciu pomocou reverzného proxy servera \emph{Apache Guacamole}, ktorý sa pripája na konzoly zariadení.

HTML5 mód web rozhrania EVE-ng bol však reagoval výrazne pomalšie pri práci s topológiou v provonaní s natívnym módom.

- predvolená vs KIS verzia integračného balíčka -> SSH tunely (pre vzdialený prístup k zariadeniam v topológii).

\section{Počítačové siete 1 a 2}

Na predmet Počitačové siete 2 bol použitý fyzický EVE-ng server.

- náhrada/doplnok pre nástroj Packet Tracer

Vypracované topológie:
- sumárne laboratórne cvičenie \emph{Packet Tracer Packet Tracer Skills Integration Challenge 8.6.1}
- Topológia s point-to-point technológiami.

\section{Projektovanie sietí 1}

náhrada/doplnok Dynamips servera (s takym vykonom, t.j. cca 10/15skupin po 10 routroch)

\section{Projektovanie sietí 2}

VMware inštalácia

V EVE-ng boli úspešne dokončené semestrálne práce s využitím technológii VPLS a Seamless MPLS.

\section{CCNP Routing}

\section{CCNP Switching}