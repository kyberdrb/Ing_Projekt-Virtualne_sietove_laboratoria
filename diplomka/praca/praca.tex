\documentclass[a4paper,12pt]{report}
\usepackage[slovak,english]{babel}
\usepackage[T1]{fontenc}             
\usepackage[utf8]{inputenc}    
\usepackage{lmodern}
\usepackage{amsmath}
\usepackage{amssymb,amsfonts,amscd}
\usepackage{array,hhline}
\usepackage{makeidx}
\usepackage{fancyhdr}
\usepackage{graphicx}
\usepackage{listings}
    %%\usepackage{titlepage}
    %%\usepackage{multicol}
\usepackage{eurosym}
\usepackage{url,mathptmx} 
\usepackage[pdftex,unicode,bookmarks=false]{hyperref}

% Celkovy pocet stran zaverecnej prace
% Uzitocny v abstrakte
\usepackage{lastpage}

\renewcommand{\baselinestretch}{1.5}  % pre zvascenie riadkovania

\addtolength{\oddsidemargin}{-.5cm}
\addtolength{\evensidemargin}{-2.9cm}   
\addtolength{\topmargin}{0cm}
\addtolength{\textheight}{0pt}
\addtolength{\textwidth}{2.cm}
\addtolength{\textheight}{2.cm}
\newlength{\verbcorr}
\setlength{\verbcorr}{0ex}

\graphicspath{{./obrazky/}}

\newcommand{\nazovpraceSK}{Analýza nástrojov na virtualizáciu sieťových prvkov a ich použitie vo vyučovacom procese}
\newcommand{\nazovpraceEN}{Analysis of the tools for network devices virtualization and their use in learning process}

%%%%%%%%%%%%%%%%%%%%%%%%%%%%%%%%%%%%%%%%%%%%%%%%%

\begin{document}
\selectlanguage{slovak}

% Uvodne strany sa necisluju
% Cisluje sa az od obsahu
\pagestyle{empty}

\begin{titlepage}

\phantom.

\bigskip

\begin{center}
{\sc\LARGE Žilinská Univerzita v Žiline}

\medskip

{\sc\Large Fakulta riadenia a informatiky}

\vspace{4cm}

{\sc\LARGE Diplomová práca}

\medskip

{\large\bf \nazovpraceSK}

\medskip

\end{center}

\phantom.\hfill
%\begin{minipage}{10cm}
\begin{center}

\vspace*{\fill}

\begin{tabu} to 1.0 \textwidth { X[l] X[r] }
 28360320182322 &  \\
 Žilina 2018  & Bc. Andrej Šišila  \\
\end{tabu}

\end{center}
%\end{minipage}
\hspace{1.7cm}\phantom.

\vspace{2.9cm}

\phantom.
\end{titlepage}

\newpage
\begin{titlepage}

\phantom.

\bigskip

\begin{center}
{\sc\LARGE Žilinská Univerzita v Žiline}

\medskip

{\sc\Large Fakulta riadenia a informatiky}

\vspace{4cm}

{\sc\LARGE Diplomová práca}

\medskip

{\large\bf \nazovpraceSK}

\end{center}

\phantom.\hfill
%\begin{minipage}{10cm}
\begin{center}


\begin{tabu} to 1.0 \textwidth { X[4,l] X[8,r] }
 Študijný odbor: & číslo št. program ASI? Aplikované sieťové inžinierstvo \\ 
 Katedra: & Katedra informačných sietí \\
 Vedúci diplomovej práce: & doc. Ing. Pavel Segeč, PhD. \\
\end{tabu}

\vspace*{\fill}

\begin{tabu} to 1.0 \textwidth { X[l] X[r] }
 28360320182322 &  \\
 Žilina 2018  & Bc. Andrej Šišila  \\
\end{tabu}

\end{center}
%\end{minipage}
\hspace{1.7cm}\phantom.

\vspace{2.9cm}

\phantom.
\end{titlepage}

\newpage
\begin{huge}
    TODO - TU PÔJDE ZADANIE DIPLOMOVEJ PRÁCE
\end{huge}

\newpage
\chapter*{Poďakovanie}
\thispagestyle{empty}

Chcel by som sa poďakovať doc. Ing. Pavlovi Segečovi, PhD. za aktívne vedenie a usmerňovanie projektu, Ing. Marekovi Moravčíkovi a Ing. Jakubovi Hrabovskému za pomoc pri riešení technických problémov, Bc. Radovanovi Kyjakovi a Bc. Radovanovi Kohutiarovi za asistenciu pri vypracovávaní projektu a Mgr. Jane Uramovej, PhD, Bc. Dušanovi Vágnerovi, Bc. Jakubovi Stehlíkovi a Bc. Marekovi Brodecovi za spätnú väzbu k projektu a jeho testovanie vo vyučovacom procese.

\newpage
\begin{abstract}

\noindent
{\sc Bc. Šišila Andrej:} {\em \nazovpraceSK} [Diplomová práca] 

\noindent
Žilinská Univerzita v Žiline, Fakulta riadenia a informatiky, Katedra informačných sietí.

\noindent  
Vedúci: doc. Ing. Pavel Segeč, PhD.
 
\noindent  
Stupeň odbornej kvalifikácie: Inžinier v odbore Aplikované sieťové inžinierstvo, Žilina. 

\noindent
FRI ŽU v Žiline, 2017 s. \pageref{LastPage}

\bigskip

Obsahom práce je nasadenie riešenia virtuálneho sieťového laboratória do vyučovacieho procesu na Katedre informačných sietí.

V prvej časti sa zaoberáme nástrojmi pre sieťovú virtualizáciu, ktoré následne porovnávame podľa zvolených kritérii, na základe ktorých bude vybraný konkrétny nástroj.

V druhej časti bude opísaná inštalácia vybraného nástroja a úpravy, ktoré rozširovali jeho funkcie a opravovali nedostatky. Nakoniec bude opísaný aj spôsob administrácie servera.

Tretia časť je venovaná analýze vyučovaných tém pre vybrané predmety na Katedre informačných sietí.

Štvrtá časť pojednáva o získavaní a testovaní virtuálnych zariadení. Na základe testovania sa vyberú vhodné zariadenia pre vyučované predmety.

V poslednej časti bude popísane nasadzovanie nástroja do vyučovacieho procesu pre konkrétne topológie vyučované na vybraných predmetoch na Katedre informačných sietí.
\\
Kľúčové slová: virtualizácia, laboratórium, EVE-ng, GNS3, KVM, Linux

\end{abstract}

\newpage
\selectlanguage{english}
\begin{abstract}

\noindent
{\sc Bc. Šišila Andrej:} {\em \nazovpraceEN} [Diploma thesis] 

\noindent
University of Žilina, Faculty of Management Science and Informatics, Department of information networks.
 
\noindent
Tutor:  doc. Ing. Pavel Segeč, PhD.
 
\noindent
Qualification level: Engineer in field Applied network engineering, Žilina: 

\noindent
FRI ŽU v Žiline, 2017 p. \pageref{LastPage}

\bigskip

\begin{huge}
    TODO - fill abstract and keywords
\end{huge}

The main idea of this thesis is ...

Keywords: 

\end{abstract}

\selectlanguage{slovak}

\newpage
\centerline{\bf Prehlásenie}

\vspace{2em}

\noindent
Prehlasujem, že som túto prácu napísal samostatne a že som uviedol 
všetky použité pramene a literatúru, z ktorých som čerpal. 

\vspace{2em}

\noindent

\begin{huge}
    TODO - doplniť správny dátum
\end{huge}

V Žiline, dňa XX.YY.ZZZZ 

\hfill

Bc. Andrej Šišila

{\setlength{\parskip}{1pt plus 1pt}

\markboth{}{}

\newpage                % newpage je nutny pre spravne cislovanie
\pagenumbering{arabic}
\pagestyle{plain}       % Aktivuj cislovanie
\setcounter{page}{7}    % Nastav spravne cislo strany
\addcontentsline{toc}{chapter}{Obsah}
\tableofcontents

%\vspace{0pt plus 2cm}

\newpage
\addcontentsline{toc}{chapter}{Zoznam ilustrácii}
\listoffigures

%\vspace{0pt plus 2cm}

\newpage
\addcontentsline{toc}{chapter}{Zoznam tabuliek}
\listoftables
}

\markboth{}{}

\clearpage

%%%%%%%%%%%%%%%%%% obsah - koniec

%%%%%%%%%%%%%%%%%% kapitoly

\chapter*{Úvod}
\addcontentsline{toc}{chapter}{Úvod}

Virtualizácia sa stáva vo svete čoraz populárnejšou. Využíva sa v rôznych oblastiach napríklad v tzv. Cloud Computing alebo Software Defined Networking. Virtualizáciou sa zaberá aj Katedra informačných sietí Žilinskej univerzity v Žiline. Jedným z projektov, kde sa virtualizácia využíva, je projekt virtuálneho sieťového laboratória. 

Našou úlohou je preskúmať existujúce riešenia v oblasti virtuálnych sieťových laboratórii a porovnať ich podľa vopred stanovených kritérií (kapitola \ref{chap:nastroje_pre_siet_virt}). Na základe toho z nich vyberieme jedno riešenie, ktorým sa budeme v práci ďalej zaoberať. Pre vybrané sieťové laboratórium následne vypracujeme návody na inštaláciu, úpravu, používanie a nasadenie do infraštruktúry katedry (kapitola \ref{chap:eve_ng}). Následne analyzujeme kompatibilitu nástroja s rôznymi zariadeniami a vybrané zariadenia otestujeme (kapitola \ref{chap:virt_zariadenia}). Následne si vyberieme predmety, na ktoré má byť tento nástroj použítý a analyzujeme technológie, ktoré sa na nich vyučujú a odhadneme, ktoré zariadenia sú pre daný predmet vhodné (kapitola \ref{chap:analyza_vyucovania}). Nak(kapitola \ref{chap:nasadenie_do_vyucovania}).

Virtuálne sieťové laboratórium bude nástrojom, ktorý zefektívni a skvalitní výučbu sieťových technológii vo vybraných predmetoch na Katedre informačných sietí.

Téma diplomovej práce bola vybraná predovšetkým preto, aby bola dosiahnutá kvalitnejší a efektívnejší vyučovací proces na katedre. Tak budú mať učitelia aj študenti jednotnú platformu pre vyučovanie, ktorá umožní jednoducho vytvárať modelové situácie.
\chapter{Ciele práce}

Primárnym cieľom práce je nasadenie virtuálneho sieťového laboratória do vyučovacieho procesu katedry. Na naplnenie tohto cieľa bolo potrebné vykonať nasledovné úlohy:

\begin{itemize}[noitemsep]
    \item Prieskum existujúcich riešení pre virtuálne sieťové laboratórium a ich následné porovnanie na základe zvolených kritérií.
    \item Voľba konkrétneho riešenia pre virtuálne sieťové laboratórium, vyplývajúca z porovnania existujúcich riešení.
    \item Inštalácia virtuálneho laboratória infraštruktúry Katedry informačných sietí a jeho následná úprava.
    \item Výber predmetov, pre ktoré bude virtuálne laboratórium použité.
    \item Analýza vyučovaných tém na vybraných predmetoch.
    \item Testovanie zariadení pre virtuálne sieťové laboratórium.
    \item Výber vhodných zariadení pre konkrétne predmety na základe podporovaných technológii daného zariadenia.
    \item Overenie funkčnosti virtuálneho sieťového laboratória vo vyučovacom procese Katedry informačných sietí na konkrétnych predmetoch.
\end{itemize}
\chapter{Nástroje pre sieťovú virtualizáciu}
\label{chap:nastroje_pre_siet_virt}

V tejto kapitole uvádzam prehľad momentálne dostupných nástrojov virtuálnych sieťových laboratórii. Používaniu jednotlivých riešení sa budem venovať v kapitole \ref{chap:analyza_vyucovania} - \nameref{chap:analyza_vyucovania}.
  
\section{Porovnávacie kritériá}

Pri porovnávaní jednotlivých virtuálnych sieťových laboratórii som sa rozhodoval podľa nasledovných kritérii:
\begin{itemize}
    \item Použité vývojové technológie
    \item Podpora zariadení
    \item Typ používateľského rozhrania
    \item Prideľovanie portových čísel zariadeniam
    \item Vzdialený prístup ku zariadeniam (telnet, vnc, rdp)
    \item Vytvorenie/úprava/uloženie/odstránenie topológie
    \item Počet topológii, ktoré môže mať jeden používateľ spustených
    \item Možnosť práce viac ľudí naraz na rovnakom projekte
    \item Možnosť prepojiť topológiu so živou sieťou
    \item Vývoj nástroja v budúcnosti
    \item Vybrané výhody a nevýhody nástroja
\end{itemize}


\section{Dostupné riešenia}

\subsection{Cisco Packet Tracer}

Cisco Packet Tracer je, ako je zrejmé z názvu, nástroj na vizualizáciu sietí vyvíjaný spoločnosťou Cisco. Je vhodný na uvedenie do problematiky sieťových technológii. Nevýhodou je, že nie je open-source a je prístupný výlučne pre členov \emph{Cisco Networking Academy}. Na druhej strane je vyvíjaný pre platformy Windows a Linux, po emulácii aj na macOS \cite{packet_tracer_mac}. Ďalšou výhodou je podpora mobilných platforiem prostredníctvom aplikácie \emph{Packet Tracer Mobile}. Slúži na emuláciu jednoduchých aktívnych aj pasívnych sieťových prvkov a jednoduchých koncových zariadení \cite{packet_tracer}. Čo sa týka smerovačov a prepínačov, tieto sú podporované, ale iba Cisco s obmedzenou funkcionalitou. Nie je možné ho ďalej rozširovať ani funkcionálne, ani ďalšími zariadeniami napr. o zariadenia iných výrobcov alebo koncové stanice Linux/Windows.

Nástroj Packet Tracer je možné používať výlučne lokálne, pretože pre tento nástroj neexistuje serverové riešenie. Vzdialený prístup a vytváranie topológii sa realizuje prostredníctvom grafického rozhrania aplikácie. Topológie si spravuje sám používateľ aplikácie. Nástroj nevie rozlišovať rôzne typy používateľov. Je nenáročný na systémové zdroje. Umožňuje pracovať súčasne iba s jednou topológiou, ktorú je možné kedykoľvek zavrieť. Topológiu nástroj nie neumožňuje prepojiť so živou sieťou.

Na obrázku \ref{obr:packet_tracer} je znázornený nástroj Dynagen.

\begin{figure}
    \centering
    \includegraphics[width=0.75\textwidth]{packet_tracer}
    \caption{Nástroj Cisco Packet Tracer spustený v prostredí Windows}
    \cite{obr_packet_tracer}
    \label{obr:packet_tracer}
\end{figure}

\subsection{Dynamips/Dynagen}

Dynamips je open-source emulátor Cisco smerovačov na Linux/Windows \cite{dynamips}. Nástroj je v prevažnej mierie napísaný v jazyku C \cite{dynamips_github}. Podporuje iba výlučne vybrané Cisco smerovače \cite{dynamips}. Ovláda sa cez príkazový riadok. Portové čísla na vzdialený prístup sa zariadeniam prideľujú manuálne. Vzdialený prístup k zariadeniam v topológii je realizovaný protokolom \emph{telnet}. Na vytváranie topológii sa používa jednoduchý značkovací jazyk.

Nástroj Dynagen, ktorý slúži ako nadstavba nad platformou Dynamips, slúži na jednoduchšiu prácu s topológiami \cite{dynamips}. Topológie môže spravovať výlučne administrátor, pretože ani Dynamips, ani Dynagen nevedia rozlišovať rôzne typy používateľov. Počet topológii, ktoré môžu byť súčasne spustené je obmedzené iba výkonom servera. Na jednej topológii môžu pracovať aj viacerí študenti, tým že sa rozdelia portové čísla zariadení v topológii medzi študentov. Nástroj Dynamips umožňuje prepojiť topológiu so živou sieťou \cite{dynamips, dynamips_nil}. 

V súčasnosti sa o nástroj starajú vývojári nástroja GNS3 \cite{dynamips_github}. Na obrázku \ref{obr:dynamips_dynagen} je znázornený nástroj Dynagen.

\begin{figure}
    \centering
    \includegraphics[width=0.75\textwidth]{dynamips_dynagen}
    \caption{Nástroj Dynagen spustený v prostredí Windows} \cite{obr_dynamips_dynagen}
    \label{obr:dynamips_dynagen}
\end{figure}

\subsection{WEB-IOU}

WEB-IOU, je simulačný nástroj pre platformu Linux, ktorý podporuje výlučne Cisco IOU platformu. Jeho hlavnou výhodou je podpora Cisco prepínačov, ktorá pri Dynamips/Dynagen chýba. Jeho autorom je Andrea Dainese \cite{webiou_github, webiou_unetlab_unetlabv2}.Nástroj je v prevažnej mierie napísaný v jazykoch PHP a JavaScript \cite{webiou_github}. Podporuje iba výlučne vybrané Cisco smerovače na platforme IOU - IOS on Unix \cite{webiou_firewall_cx}. Je vhodný na trénovanie pri certifikáciách CCNP a do istej miery aj CCIE. 

Spravuje sa cez príkazový riadok. Používateľovi je dostupné web rozhranie. Portové čísla na vzdialený prístup sa zariadeniam prideľujú automaticky. Vzdialený prístup k zariadeniam v topológii je realizovaný protokolom \emph{telnet}. Na vytváranie topológii sa používa jednoduchý značkovací jazyk \emph{NETMAP}. Topológie môže ktokoľvek, kto má prístup k web rozhraniu, pretože ani tento nástroj nevie rozlišovať rôzne typy používateľov. Počet topológii, ktoré môžu byť súčasne spustené je obmedzené iba výkonom servera. Napriek tomu môžu na jednej topológii môžu pracovať aj viacerí študenti rovnakým spôsobom, ako pri nástroji Dynamips/Dynalab. Nástroj WEB-IOU tiež umožňuje prepojiť topológiu so živou sieťou \cite{webiou_real_network}. 

V súčasnosti sa už nástroj nevyvíja. Na obrázku \ref{obr:webiou} je znázornené webové rozhranie nástroja WEB-IOU.

\begin{figure}
    \centering
    \includegraphics[width=0.75\textwidth]{webiou}
    \caption{Webové rozhranie nástroja WEB-IOU} \cite{obr_webiou}
    \label{obr:webiou}
\end{figure}

\subsection{Cisco VIRL}

VIRL, Virtual Internet Routing Lab, je komerčný simulačný nástroj sietí vyvíjaný spoločnosťou Cisco. Podporuje nielen Cisco smerovače a prepínače, ale aj zariadenia iných výrobcov, hoci ich integrácia nemusí byť jednoduchá. Výhodou oproti iným nástrojom je možnosť pridať podporované zariadenia do topológie ako LXC kontajner. Nevýhodou je, že nepodporuje Dynamips/Dynagen emuláciu, takže na ňom nie je možné využiť existujúce virtuálne zariadenia na katedre. Nástroj je postavený na platforme Linux (Debian) a je dostupný ako virtuálny stroj pre rôzne platformy. Je vhodný na trénovanie pri certifikáciách CCNP a do istej miery aj CCIE \cite{virl_cisco}. 

Spravuje sa cez príkazový riadok. Používateľovi je dostupné web rozhranie. Portové čísla na vzdialený prístup sa zariadeniam prideľujú automaticky \cite{virl_interfacett_1}. Vzdialený prístup k zariadeniam v topológii je realizovaný protokolmi \emph{telnet} a \emph{ssh}, po úprave aj \emph{vnc} \cite{virl_ciscoskills, virl_speaknetworks}. Na vytváranie topológii sa používa Nástroj \emph{VM Maestro}. Ten poskytuje možnosť, vopred si nakonfigurovať zariadenie podľa zvolených scenárov pomocou funkcie \emph{AutoNetkit}. Aj napriek tomu, že VIRL poskytuje pomerne podrobné možnosti na konfiguráciu zariadení a topológii, jeho používanie je pomerne obtiažne, hlavne pri vytváraní topológii \cite{virl_interfacett_1, virl_interfacett_2}. Cisco VIRL vie rozlišovať rôzne typy používateľov \cite{virl_cisco_features}. Počet topológii, ktoré môžu byť súčasne spustené je obmedzené iba výkonom servera. Napriek tomu môžu na jednej topológii môžu pracovať aj viacerí študenti rovnakým spôsobom, ako pri nástroji Dynamips/Dynalab \cite{virl_interfacett_2}. Nástroj tiež umožňuje prepojiť topológiu so živou sieťou \cite{virl_speaknetworks}. 

Nástroj v súčasnosti existuje iba vo verzii \emph{Personal Edition} s licenciou na 20 zariadení, čo výrazným spôsobom obmedzuje jeho využitie pri vyučovaní. V minulosti existovali aj verzie \emph{Personal Edition} s licenciou na 30 zariadení a \emph{Academic Edition}. Rozdiel medzi Personal a Academic Edition bol iba ten, že Academic Edition bol prístupný učiteľom a študentom za výhodnejšiu cenu. Podporované funkcionality boli zhodné v oboch verziách \cite{virl_edition_differences}. Na obrázkoch \ref{obr:virl_vmmaestro} a \ref{obr:virl_web} je v tomto poradí znázornený nástroj na vytváranie topológii VM Maestro a webové rozhranie Cisco VIRL.

\begin{figure}
    \centering
    \includegraphics[width=0.75\textwidth]{virl_vmmaestro}
    \caption{VM Maestro} \cite{obr_virl_vmmaestro}
    \label{obr:virl_vmmaestro}
\end{figure}

\begin{figure}
    \centering
    \includegraphics[width=0.75\textwidth]{virl_web}
    \caption{Cisco VIRL web rozhranie} \cite{obr_virl_web}
    \label{obr:virl_web}
\end{figure}

\subsection{ViRo v2}

ViRo, resp. ViRo v2, je virtuálne laboratórium vytvorený na Katedre informačných sietí. Nástroj vznikol ako výsledok diplomovej práce Ing. Petra Hadača, pričom pokračoval v predchádzajúcej verzii nástroja, ViRo v1. Nástroj je postavený na platforme Linux. Využíva technológie tzv. \emph{LAMP Stack} servera: Linux, Apache, MySQL, PHP (Drupal). Využíva virtualizáciu pomocou QEMU/KVM a Dynamips. Jeho hlavnou výhodou je možnosť rezervovať si topológiu. Potenciálnou nevýhodou je, že nepodporuje platformu Cisco IOL. Ďalšou možnou nevýhodou je, že stavia na platforme Drupal, ktorého popularita je pomerne nízka \cite{stackoverflow_survey}.

Spravuje sa prostredníctvom web rozhrania, SSH alebo VNC prístupu. Používateľom s rolou \emph{učiteľ} a \emph{študent} je dostupné webové rozhranie. Portové čísla na vzdialený prístup sa zariadeniam dajú nastaviť manuálne. Vzdialený prístup k zariadeniam v topológii je realizovaný viacerými spôsobmi: pomocou nástroja \emph{virsh}, aplikáciou \emph{Virtual Machine Manager} prístupnou cez \emph{vnc}, \emph{noVNC} serverom alebo SSH tunelom. Vytváranie topológii a správu zariadení sa používa grafický nástroj \emph{Virtual Machine Manager}. Topológie môže vytvárať iba používateľ s rolou  \emph{učiteľ} alebo \emph{administrátor}, keďže nástroj ViRo vie rozlišovať rôzne typy používateľov. Počet topológii, ktoré môžu byť súčasne spustené je obmedzené iba výkonom servera. Napriek tomu môžu na jednej topológii môžu pracovať aj viacerí študenti rovnakým spôsobom, ako pri nástroji Dynamips/Dynalab. Nástroj umožňuje prepojiť topológiu so živou sieťou pomocou \emph{bridge} rozhrania \cite{viro_hadac}.

\subsection{UNetLab}

UNetLab, Unified Networking Lab, skrátene UNL, je open-source  simulačný nástroj pre platformu Linux, ktorý integruje všetky vyššie uvedené technológie najednom mieste: Dynamips, Cisco IOU aj zariadenia tretích strán (QEMU). Nástroj je postavený na platforme Linux. Jeho autorom je Andrea Dainese. V prevažnej mierie je napísaný v jazykoch PHP a JavaScript \cite{webiou_unetlab_unetlabv2, unetlab_github}. Je vhodný nielen na trénovanie pri Cisco certifikáciách, ale aj na testovanie kompatibility rôznych výrobcov.

Spravuje sa cez príkazový riadok. Používateľovi je dostupné web rozhranie. Portové čísla na vzdialený prístup sa zariadeniam prideľujú automaticky. Vzdialený prístup k zariadeniam v topológii je realizovaný protokolom \emph{telnet} alebo \emph{vnc}. Topológie sa vytvárajú vo webovom rozhraní prepájaním uzlov medzi sebou pomocou myši, pričom sa na pozadí sa generuje súbor v značkovacom jazyku \emph{NETMAP}. Topológie môže ktokoľvek, kto má prístup k web rozhraniu, pretože ani tento nástroj nevie rozlišovať rôzne typy používateľov, hoci v istej verzii nástroja táto funkcia bola podporovaná \cite{unetlab_github}. Počet topológii, ktoré môžu byť súčasne spustené je obmedzené iba výkonom servera. Napriek tomu môžu na jednej topológii môžu pracovať aj viacerí študenti rovnakým spôsobom, ako pri nástroji Dynamips/Dynalab. Jeden používateľ môže mať otvorenú práve jednu topológiu. Topológia sa dá zatvoriť až vtedy, keď v nej nie sú spustené žiadne zariadenia. Nástroj UNetLab tiež umožňuje prepojiť topológiu so živou sieťou pomocou \emph{bridge} rozhrania \cite{webiou_real_network}.

Vývoj tohto nástroja bol zastavený. UNetLab ďalej vyvíjala iná skupina vývojárov, ktorý nástroj premenovala na EVE-ng a migrovala ho z platformy Ubuntu 14.04 na Ubuntu 16.04. Jeho pôvodný autor následne začal s vývojom ďalšej verzie nástroja UNetLab, UNetLabv2. Na obrázku \ref{obr:unetlab_web} je znázornené webové rozhranie nástroja UNetLab.

\begin{figure}
    \centering
    \includegraphics[width=0.75\textwidth]{unetlab_web}
    \caption{Webové rozhranie nástroja UNetLab}
    \cite{obr_unetlab_web}
    \label{obr:unetlab_web}
\end{figure}

\subsection{EVE-ng}
\label{chap:virt_lab_eve_ng}

EVE-ng je simulačný nástroj sietí, ktorý vznikol ako klon a nasledovník nástroja UNetLab. Celkovou funkcionalitou, až na niektoré zmeny, napr. použitie MySQL namiesto SQLite, pridaná podpora pre ďalšie zariadenia), a vzhľadom webového rozhrania sa preto veľmi podobá na svojho predchodcu. Nástroj je postavený na platforme Linux a vyvíjaný prevažne v jazykoch JavaScript a PHP. Web rozhranie je realizované ako webová aplikácia s použitím framework nástroja \emph{Angular JS} a \emph{Twitter Bootstrap} \cite{eve_ng_technologies}.

EVE-ng sa v priebehu marca 2018 rozdelilo na tri verzie: Community, Professional a Learning Centre. Community verzia je open-source, aj keď \emph{gitlab} repozitár bol neprístupný pre verejnosť v priebehu novembra/decembra 2017. Napriek tomu sú na serveri všetky súbory prístupné a upravovateľné. Túto verziu je možné slobodne šíriť a upravovať. Verzia Professional obsahuje niektoré funkcionality, ktoré uľahčujú prácu s nástrojom, ale vývojári sa rozhodli spoplatniť ju. Learning Centre verzia obsahuje funkcionality na nasadenie do produkčného prostredia, ako je napr. rozdelenie používateľov do používateľských rolí. Podrobný zoznam podporovaných funkcii v jednotlivých verziách je dostupný v \cite{eve_ng_versions_table} a \cite{eve_ng_versions_list}.

Vzhľad webového rozhrania EVE-ng je znázornený na obrázku \ref{obr:eve_ng_web}. Rozdiely jednotlivých verzii EVE-ng sú znázornené v tabuľke \ref{tab:eve_ng_versions}.

\begin{figure}
    \centering
    \includegraphics[width=0.75\textwidth]{eve_ng_web}
    \caption{Webové rozhranie nástroja EVE-ng}
    \label{obr:eve_ng_web}
\end{figure}

\begin{longtable}{| m{3cm} | m{2cm} | m{2cm} | m{2cm} | m{4cm} |}
\caption{Porovnanie EVE-ng verzii}
\cite{eve_ng_versions_table}
\label{tab:eve_ng_versions} \\
\hline
Features/Edition                                      & Community         & Professional    & Learning Center      & Description                                                                                                   \\ \hline
Price                                                 & Free              & 99 EUR w/o VAT  & 99 EUR + Added Roles &                                                                                                               \\ \hline
User's roles                                          & admin only        & admin only      & admin, user, editor  & Restrictions of the EVE usage, WEB UI, per user based                                                         \\ \hline
Lock user per folder                                  & No                & No              & Yes                  & User cannot see other EVE folders, only his own                                                               \\ \hline
Lock user edit rights                                 & No                & No              & Yes                  & User cannot edit labs, images etc                                                                             \\ \hline
Shared Lab Folder                                     & No                & No              & Yes                  & Shared lab folder visible for all users                                                                       \\ \hline
User's account validity ( 1/4 Hour accuracy )         & No                & No              & Yes                  & Ability to set calendar validity for account, Date and time ( From -\textgreater To )                         \\ \hline
Lab Timer                                             & No                & Yes             & Yes                  & Timer for Lab training                                                                                        \\ \hline
Running labs folder                                   & No                & Yes             & Yes                  & User can run more than one lab. Running labs will appear in special running labs folder. Per user based       \\ \hline
Node limit per lab                                    & 63                & 1024            & 1024                 & Limit of nodes to run per lab                                                                                 \\ \hline
TCP ports                                             & fixed 128 per POD & Dynamic 1-65000 & Dynamic 1-65000      & Automatic TCP port choose for telnet session                                                                  \\ \hline
Local Wireshark capture                               & Yes               & No              & No                   & Local wrapper using ssh and root password to the EVE                                                          \\ \hline
Local Telnet client                                   & Yes               & Yes             & Yes                  & Local wrapper using locally installed telnet client                                                           \\ \hline
Local VNC client                                      & Yes               & Yes             & Yes                  & Local wrapper using locally installed vnc client                                                              \\ \hline
Wireshark integrated                                  & No                & Yes             & Yes                  & Docker integrated wireshark                                                                                   \\ \hline
Docker container support                              & No                & Yes             & Yes                  & Docker container support                                                                                      \\ \hline
Running nodes interface connections (hot connections) & No                & Yes             & Yes                  & Hot/live nodes interface connection                                                                           \\ \hline
NAT Cloud                                             & No                & Yes             & Yes                  & Integrated NAT cloud, connect node to the internet. NAT to the EVE management interface DHCP 169.254.254.0/24 \\ \hline
HTML console without Wireshark capture                & Yes               & No              & No                   & HTML console                                                                                                  \\ \hline
HTML console with Wireshark capture                   & No                & Yes             & Yes                  & HTML wireshark capture                                                                                        \\ \hline
HTML Desktop Console                                  & No                & Yes             & Yes                  & Integrated Docker PC management                                                                               \\ \hline
Multi startup configuration choose per lab            & No                & Yes             & Yes                  & Option to create and boot lab from different startup configurations, multi startup config                     \\ \hline
Export/Import configs or config packs to local PC     & No                & Yes             & Yes                  & Option import and export single config or config packs to the lab                                             \\ \hline  
\end{longtable}

Z tabuľky vyplýva, že verzie \emph{Community} je ako jediná bezplatná. Najväčšími výhodami ostatných verzii je rozdelenie používateľov do používateľských rolí (iba v \emph{Learning Centre}), zatvorenie topológie s už spustenými zariadeniami, prehľad zatvorených topológii so spustenými zariadeniami a zvýšený limit na počet spustených zariadení pre topológiu.

Doplnením a rozšírením funkcionality EVE-ng Community verzie je venovaná kapitola \ref{chap:eve_ng_uprava_zdroj_kodov} - \nameref{chap:eve_ng_uprava_zdroj_kodov}

\subsection{GNS3}

GNS3, Graphical Network Simulator 3, je open-source sieťový simulátor sietí. Integruje všetky virtualizačné technológie najednom mieste: Dynamips, Cisco IOU aj zariadenia tretích strán (QEMU). Od verzie 1.5 sú v GNS3 podporované aj Docker kontajnery, čo je veľkou výhodou oproti iným nástrojom, pretože Docker kontajnery potrebujú menej systémových prostriedkov \cite{gns3_docker}.

GNS3 sa skladá z klientskej a serverovej časti. Klientská časť pozostáva z aplikácie \emph{GNS3 Client} a je celá napísaná v jazyku Python \cite{gns3_gui_github}. Klientská aplikácia je multiplatformová t.j. je kompatibilná s platformami Windows, Linux a macOS. Existuje aj klientská webová aplikácia \emph{gns3-web} \cite{gns3_web_github}.

Serverová časť môže byť realizovaná ako serverová aplikácia \emph{GNS3 Server}, ako virtuálny stroj \emph{GNS3 VM} alebo ako vzdialený server.

Serverová aplikácia \emph{GNS3 Server} sa spustí predvolene pri spustení klientskej aplikácie. Rovnako, ako GNS3 klientská aplikácia, aj serverová aplikácia je napísaná celá v jazyku Python \cite{gns3_server_github}.

GNS3 VM aj vzdialený server je postavený na platforme Linux. Vzdialený server nemusí nutne byť fyzický server, na ktorom je nasadený GNS3 server. Môže byť v ľubovoľnom virtualizačnom nástroji, napr. vo VMware. VMware je odporúčaná voľba pre tento virtualizačný nástroj, pretože podporuje vnorenú virtualizáciu, čo VirtualBox doposiaľ nepodporuje \cite{nested_virtualization}.

GNS3 VM resp. vzdialený server sa spravuje cez príkazový riadok. Používateľovi je dostupná klientská aplikácia. Portové čísla na vzdialený prístup sa zariadeniam prideľujú automaticky, avšak je možné manuálne meniť rozsah, v akom sa majú portové čísla automaticky prideľovať, dokonca umožňuje aj manuálnu zmenu čísla portu pre jednotlivé zariadenia \cite{gns3_console_ports, gns3_console_ports_remote}. Vzdialený prístup k zariadeniam v topológii je realizovaný protokolom \emph{telnet}, \emph{vnc} alebo\emph{rdp}. Topológie sa vytvárajú v klientskej aplikácii prepájaním uzlov medzi sebou pomocou myši. V predvolenom nastavení sú všetky topológie zdieľané a môže ich meniť ktokoľvek, kto má prístup k web rozhraniu, pretože v predvolenom nastavení nástroj nevie rozlišovať rôzne typy používateľov ani ich izolovať. Počet topológii, ktoré môžu byť súčasne spustené je obmedzené iba výkonom servera. Napriek tomu môžu na jednej topológii môžu pracovať aj viacerí študenti tým, že si otvoria rovnaký projekt na vzdialenom serveri. Zmeny v takejto zdieľanej topológii sa prejavia okamžite všetkým používateľom. Jeden používateľ môže mať spustených aj viacero topológii, v klientská aplikácia však dovoľuje pracovať iba s jednou naraz. Topológia sa dá kedykoľvek zatvoriť. Aj GNS3, podobne ako aj ďalšie nástroje, umožňuje prepojiť topológiu so živou sieťou pomocou \emph{bridge} rozhrania.

Vývoj tohto nástroja stále pokračuje. Na obrázku \ref{obr:gns3_client} je znázornená klientská aplikácia GNS3 Client.

\begin{figure}
    \centering
    \includegraphics[width=0.75\textwidth]{gns3_client}
    \caption{Klientská aplikácia GNS3}
    \label{obr:gns3_client}
\end{figure}

\subsection{UNetLabv2}

UNetLabv2 je nasledovníkom nástroja UNetLab. Je postavený na platforme Docker kontajnerov. Jednotlivé úlohy sú distribuované naprieč kontajnermi. To zaisťuje lepšiu škálovateľnosť pri zachovaní rovnakej funkcionality. Zatiaľ ešte nie je verejne nedostupný.

Architektúra nástroja UNetLabv2 je znázornená na obrázku \ref{obr:unetlabv2_arch}

\begin{figure}
    \centering
    \includegraphics[width=0.75\textwidth]{unetlabv2_arch}
    \caption{Architektúra nástroja UNetLabv2} \cite{obr_unetlabv2_arch}
    \label{obr:unetlabv2_arch}
\end{figure}

\section{Vyhodnotenie}

Z vyššie uvedených nástrojov má zmysel zaoberať sa nástrojmi EVE-ng a GNS3 z nasledovných dôvodov:
\begin{itemize}
    \item Open-source vývoj oboch nástrojov umožňuje ich používanie bez poplatkov, obáv o porušenie licenčných podmienok a poskytuje možnosť upravovať ich podľa vlastných požiadaviek.
    \item Podpora zariadení od rôznych výrobcov.
    \item Jednoduché ovládanie.
\end{itemize}

V priebehu projektu sme sa preto zamerali na nástroje GNS3 a EVE-ng. Počas neho sa však ukázalo, že GNS3 nie je vhodná pre vzdialené použitie, preto sme sa týmto nástrojom ďalej nezaoberali. V čase skúmania bol nástroj GNS3 vo verzii 1.5.3. Keď sme skúšali použiť GNS3 ako vzdialený server, klientská aplikácia sa na GNS3 vzdialený server nevedela pripojiť, hoci sme postupovali podľa návodov na GNS3 stránke a pri testovaní nestála v pripojení na server žiadna prekážka napr. firewall.

GNS3 od vydania stabilnej verzie 2.0.0 opravila problém s nasadením ako vzdialený server. Avšak vtedy som už začal s hlbším skúmaním EVE-ng. Skúmanie dvoch nástrojov naraz do hĺbky by bolo časovo veľmi náročné. GNS3 slúžila počas skúmania ako podporný nástroj pre pochopenie rôznych technických súčastí virtualizácie sieťových zariadení.

Vo zvyšku diplomovej práce sa zaoberám nástrojom EVE-ng a jeho nasadením do vyučovania na katedre.

\chapter{EVE-ng}

\section{Inštalácia}

Kde všade je EVE-ng server nainštalovaný a v akej verzii.

\begin{itemize}
    \item VMware Workstation Player - .49
    \item Fyzický server - .50
\end{itemize}

Parametre VMware virtuálky aj fyzického servera. V akom stave je EVE-ng na oboch platformách?

Uvedené tvrdenia platia pre EVE-ng vo vydaní \emph{Community Edition} vo verzii 2.0.3-86.

  - Postup
    - Inštalácia Ubuntu Server 16.03
    - Konfigurácia Ubuntu Server
    - Inštalácia EVE-ng do Ubuntu Server
    - Konfigurácia EVE-ng
      - Obnovenie súborov a adresárov
        - Skripty
        - Zariadenia
        - Databázy
      - Automatizácia zálohovania nástrojom "cron"
      - Cisco IOL/IOU licencia
      - Zabezpečenie servera
  - úprava šablón

\section{Úpravy po inštalácii}

Tu pomôžu názvy súborov v adresári \emph{eve-ng} v repozitári/CD.

\subsection{Linux server}
    Zabezpečenie
  - Systém
  - SSH
  - Web server
  - Databázový server
  
\subsection{EVE-ng}

Ako som modifikoval zdrojový kód EVE-ng.

\subsection{Klientský počítač}

- SSH tunely (pre vzdialený prístup k zariadeniam v topológii).

\section{Administrácia}

\subsection{Adresárova štruktúra}

\subsection{Zálohovanie}

\subsection{Monitorovanie}

\subsubsection{EVE-ng - System status}

\subsubsection{netdata}

\section{Používanie}

\subsection{Správa používateľov v EVE-ng}

\chapter{Analýza vyučovania}
\label{chap:analyza_vyucovania}

Virtuálne sieťové laborarórium má byť nasadené na vybrané predmety vyučované na katedre. Na to treba analyzovať vyučované témy týchto predmetov. Na základe toho sa budú získavať a testovať zariadenia, čomu je venovaná kapitola \ref{chap:virt_zariadenia} - \nameref{chap:virt_zariadenia}.

V tejto kapitole opisujem vyučované témy týchto predmetov:

\begin{itemize}[noitemsep]
    \item Počítačové siete 1 (5BN103)
    \item Počítačové siete 2 (5BN104)
    \item Projektovanie sietí 1 (5IN116)
    \item Projektovanie sietí 2 (5IP111)
    \item CCNA Security
    \item Pokročilé prepínanie v informačno-komunikačných sieťach (5IN139)
    \item Pokročilé smerovanie v informačno-komunikačných sieťach (5IN124)
\end{itemize}

Výber predmetov ovplyvňoval fakt, že na nich vyučujú sieťové technológie. Nástroj má byť v prvom rade používaný na predmetoch, kde sa vyučujú pokročilejšie sieťové technológie t.j. Projektovanie sietí 1, Projektovanie sietí 2, Pokročilé prepínanie v informačno-komunikačných sieťach a Pokročilé smerovanie v informačno-komunikačných sieťach.

V nasledujúcich častiach budú opísané zariadenia, ktoré sa používajú pri výučbe týchto predmetov, ako aj vyučované technológie.

Vyučované technológie boli získané z informačných listov predmetov a z plánov predmetu od vyučujúcich. Zoznam vyučovaných technológii je dostupný v kapitole \ref{chap:cd} v bodoch \ref{item:zoznam_technologii_s_podporou_zariadeni} a \ref{item:zoznam_technologii_txt}




\section{Počítačové siete 1}

Predmet obsahuje témy z oblasti CCNA 2 vrátane prepínacích technológii z CCNA 3. Momentálne sa na predmete používa iba nástroj Packet Tracer.

Na predmete sa vyučuje IPv4 a IPv6 statické smerovanie, RIPv2, RIPng, SVI, STP BPDU Guard, PortFast, VLAN, VLAN Trunk 802.1Q, InterVLAN smerovanie - Router on a Stick, VTP v1/v2/v3, STP, PVST+, RPVST+, Extended VLAN, L2 EtherChannel PAgP a LACP, L3 EtherChannel PAgP a LACP, HSRP IPv4, HSRPv2 IPv4 a IPv6, VRRPv2 IPv4 VRRPv3 IPv4 a IPv6, GLBP IPv4 a IPv6, ACL IPv4 a IPv6, DHCP IPv4 a IPv6, NAT, LLDP, CDP, Syslog, NTP, SNMP, SPAN.

Na predmete sa pracuje predovšetkým so smerovačmi a prepínačmi Cisco a jednoduchými koncovými zariadeniami v rámci možností nástroja Packet Tracer. V budúcnosti sa uvažuje o integrácii Juniper smerovačov a pokročilejších koncových zariadení na platforme Linux a Windows.




\section{Počítačové siete 2}

Predmet obsahuje témy z oblasti CCNA 3 a CCNA 4 okrem prepínacích technológii. Momentálne sa na predmete používajú nástroje Packet Tracer a na niektoré topológie nástroj Dynamips/\\Dynagen. V druhom menovanom nástroji topológie pozostávajú z zariadení Cisco 2691.

Na predmete sa vyučujú témy EIGRP IPv4 a IPv6, OSPFv2 Single-Area a Multi-Area, OSPFv3 Single-Area a Multi-Area, PPP, MLPPP, HDLC, PPPoE, GRE, eBGP IPv4.

Na predmete sa pracuje predovšetkým so smerovačmi a prepínačmi Cisco a jednoduchými koncovými zariadeniami v rámci možností nástroja Packet Tracer. V budúcnosti sa uvažuje o integrácii Juniper smerovačov a pokročilejších koncových zariadení na platforme Linux a Windows.




\section{Projektovanie sietí 1}

Predmet obsahuje niektoré témy z oblasti CCNP Routing a ďalších pokročilých smerovacích technológii. Momentálne sa na predmete používa nástroj Dynamips/Dynalab. V ňom pozostávajú topológie zo zariadení Cisco 2691 a Cisco 7200.

Na predmete sa vyučujú témy OSPFv2 Multi-Area, IS-IS IPv4, IGMP v1/v2/v3, IGMP Snooping, PIM Dense Mode/Sparse Mode/Sparse-Dense Mode, PIM Any-Source Multicast, PIM Source-Specific Multicast, Manual RP, Auto-RP, BSR, Anycast RP, BGP IPv4, Router Reflector, MP-BGP,BGP mVPN, Hub \& Spoke VPN, Draft Rosen, BGP L3 VPN, MPLS, LDP, RSVP, VPLS.

Na predmete sa pracuje predovšetkým so smerovačmi a prepínačmi Cisco. Koncové zariadenia sa takmer vôbec nepoužívajú, iba ak by sa skupina rozhodla pracovať s fyzickými zariadeniami. V budúcnosti sa uvažuje o integrácii Juniper smerovačov a pokročilejších koncových zariadení na platforme Linux a Windows, hlavne pre účely vyučovania \emph{multicast} technológii.



\section{Projektovanie sietí 2}

Predmet obsahuje témy z oblasti pokročilých smerovacích technológii. Výučba tohto predmetu bola v šk. roku 2017/2018 realizovaná v nástroji EVE-ng v rámci pilotného nasadenia do vyučovania.

Na predmete sa vyučujú témy VPLS, EVPN, Seamless MPLS, BGP mVPN NG.


Na predmete sa pracovalo so smerovačmi a prepínačmi Cisco a smerovačmi Juniper a Nokia. Nástroj EVE-ng podporuje koncové zariadenia na platforme Linux a Windows a je ich možné integrovať do topológie.




\section{CCNA Security}
  
Predmet obsahuje prehľad tém a technológii z oblasti bezpečnosti v rámci linkovej, sieťovej a aplikačnej vrstvy. Výučba tohto predmetu je plánovaná na šk. roku 2018/2019 namiesto predmetu Optimalizácia konvergovaných sietí. Zvažuje sa nad jeho realizáciou v nástroji EVE-ng v rámci ďalšieho nasadenia do vyučovania.

Keďže predmet je nový a jeho osnova ešte nie je pevne stanovená, zoznam vyučovaných technológii nie je uvedený. Prehľad tém vyučovaných na kurze čerpá z materiálov Cisco Network Security (IINS) (210-260), ktorý je dostupný na stránke \cite{ccna_security_topics}.

Predmet vyžaduje Cisco zariadenia, konkrétne prepínače, smerovače, popr. Cisco firewall a koncové zariadenia na platforme Linux alebo Windows.





\section{Pokročilé prepínanie v informačno-komunikačných sieťach}

Predmet obsahuje témy z oblasti CCNP Switching. Momentálne sa na predmete používajú fyzické zariadenia, keďže katedra momentálne nedisponuje riešením na virtualizáciu prepínačov.

Vyučované témy na tomto predmete sa do veľkej miery zhodujú s predmetom Počítačové siete 1, avšak témy sú preberané podrobnejšie. Osnova predmetu obsahuje navyše témy IP SLA, STP BPDU Filter, MST, CEF, MLS, FHRP IPv4 a IPv6, NTP Authentication, Cisco ISL trunks, DHCP Snooping, PVLAN.

Na predmete sa pracuje predovšetkým s fyzickými prepínačmi a Cisco. Nástroj EVE-ng umožňuje do topológie integrovať aj rôzne Cisco prepínače.





\section{Pokročilé smerovanie v informačno-komunikačných sieťach}

Predmet obsahuje témy z oblasti CCNP Routing. Momentálne sa na predmete používa nástroj Dynampis/Dynalab. Ten obsahuje topológie, ktoré využívajú smerovač rady Cisco 7200, keďže je to jedniný Dynamips smerovač, ktorý v plnom rozsahu podporuje technológie vyučované na predmete.

Vyučované témy na tomto predmete sa do veľkej miery zhodujú s predmetom Počítačové siete 2, avšak témy sú preberané podrobnejšie. Osnova predmetu obsahuje navyše témy PBR, Route redistribution, Route filtering, IP SLA, MP-BGP.

Na predmete sa pracuje predovšetkým so smerovačmi a prepínačmi Cisco. Nástroj EVE-ng umožňuje do topológie integrovať aj rôzne Cisco smerovače, od jednoduchších, až po pokročilejšie.
\chapter{Virtuálne zariadenia}

\section{Získavanie}

Zariadenia boli získavané z rôznych zdrojov. Predovšetkým boli použité zariadenia, ktoré sa už na katedre používali.

\section{Testovanie}

\subsection{Testovanie spustiteľnosti}

Kompatibilita zariadení.
Kritériá

\subsubsection{Metodika}

\subsubsection{Vyhodnotenie}

Sumárny prehľad.ods

\subsection{Testovanie systémových požiadaviek}

Kritériá

\subsubsection{Metodika}

Testované skriptom.


Pre vybrané zariadenia boli merané tieto veličiny:
\begin{itemize}
\item vyťaženie CPU
\item nároky na operačnú pamäť
\item vyťaženie pevného disku
\end{itemize}

\subsubsection{Vyhodnotenie}

benchmarking.zip

Maximálny počet zariadení každého typu je možné odhadnúť sčítaním ich systémových požiadaviek. Rovnako odhadneme aj maximálny počet spustených topológii daného typu.

\subsection{Testovanie technológii}

Kritériá.
Zatiaľ boli testované podporované technológie iba na Cisco zariadeniach.

\subsubsection{Metodika}

Testované skriptom

\subsubsection{Vyhodnotenie}

Vyučované technológie.ods

\section{Kritériá testovania}
\chapter{Nasadenie do vyučovania}

Rozpísať, ako sa ktorý nástroj správal pri vypracovávaní úloh z daného predmetu.

Urobiť napríklad scenár: syslog a snmp server z inych hostov a routrov

\section{Počítačové siete 1 a 2}

sumárne laboratórne cvičenie “Packet Tracer Packet Tracer Skills Integration Challenge 8.6.1”

Topológia s point-to-point technológiami.

\section{Projektovanie sietí 1}

náhrada/doplnok Dynamips servera (s takym vykonom, t.j. cca 10/15skupin po 10 routroch)

\section{Projektovanie sietí 2}

\subsection{VPLS}

\subsection{Seamless MPLS}

\section{CCNP Routing}

\section{CCNP Switching}
\chapter{Záver}

Cieľom práce bolo nasadenie virtuálneho sieťového laboratória do vyučovacieho procesu katedry. Najprv bolo potrebné preskúmať existujúce riešenia pre virtuálne sieťové laboratóriá a porovnať ich na základe zvolených kritérií. Z porovnania vyplynulo, že existujú viaceré virtuálne laboratória, ktorými sa má zmysel ďalej zaoberať: GNS3 a EVE-ng. Nakoniec však bol zvolený druhý menovaný nástroj. Následne bol tento nástroj nainštalovaný na virtuálnu platformu VMware a fyzický server.

Po nasadení nástroja do infraštruktúry katedry boli analyzované vyučované témy vybraných predmetov na Katedre informačných sietí. Táto analýza pomohla pri získavaní virtuálnych zariadení do nástroja EVE-ng. Získané zariadenia boli v nástroji EVE-ng testované na ich spustiteľnosť, systémové požiadavky a podporu vyučovaných tém. Pre analyzované predmety boli vybrané zariadenia, ktoré boli počas nasadenia do vyučovania použité pri vytváraní topológii. Zariadenia boli pre predmety vyberané tak, aby pokryli čo najväčšiu časť vyučovaných tém. Študenti na cvičeniach pod vedením vyučujúceho používali tento nástroj, aby sa overila jeho použiteľnosť v reálnom vyučovacom procese.

Celý proces bol dôsledne a prehľadne dokumentovaný. Dokumentácia podrobne opisuje celý proces práce od inštalácie nástroja EVE-ng a jeho následnej úpravy až po získavanie a testovanie virtuálnych zariadení a nasadenie nástroja do vyučovania. Dokumentácia je prítomná na priloženom CD, ktorého adresárovú štruktúru je možné vidieť v kapitole \ref{chap:cd}.

Na výsledkoch tejto práce sa dá v budúcnosti pokračovať v mnohých smeroch. Práca môže slúžiť ako východiskový bod pri skúmaní ďalších virtuálnych sieťových laboratórií s použitím metodík popísaných v rôznych častiach tejto práce. To by mohlo viesť k vytvoreniu ideálneho riešenia pre virtuálne sieťové laboratórium, ktoré by v sebe kombinovalo výhody nástrojov ViRo2, GNS3 a EVE-ng, pričom jadro by tvoril práve nástroj GNS3. Ideálne sieťové laboratórium by obsahovalo rezervácie topológii z nástroja ViRo2, grafické používateľské rozhranie, jednoduchú škálovateľnosť naprieč viacerými servermi, dodatočnú podporu Docker popr. LXC/LXD kontajnerov a izoláciu používateľov spolu ich súbormi a adresármi. Vo svojej práci som sa snažil splniť aspoň niektoré zo spomenutého stručného zoznamu požiadaviek.

Ďalším krokom v pokračovaní projektu by mohlo byť nasadenie nástroja EVE-ng resp. GNS3 do LXC kontajnera pre lepšiu škálovateľnosť. Takisto by bolo zaujímavé preskúmať využitie katedrového OpenStack riešenia ako virtuálne sieťové laboratórium.

V každom prípade nasadenie konkrétneho nástroja do vyučovacieho procesu katedry posunulo vyučovanie na nej na vyššiu úroveň, čo dokazuje, že virtuálne sieťové laboratórium tvorí dôležitú súčasť pri vyučovaní sieťových technológii.
\chapter{Prílohy}

CD médium obsahuje:

\begin{itemize}
    \item Text bakalárskej práce,
    \item Návody na používanie pre virtualizačný nástroj
    \item Tabuľkový dokument so sumárnym vyhodnotením všetkých otestovaných zariadení pre virtualizačný nástroj
    \item Tabuľkový dokument s prehľadom všetkých vyučovaných technológii na vybraných predmetoch a ich kompatibilitou s vybranými virtuálnymi zariadeniami
\end{itemize}

%%%%%%%%%%%%%%%%%% literatura

%%%%%%%%%%%%%%%%%%%%%%%%%%%%%%%%%%%%%%%%%%%%%%%%%%%%%%%%%%%%%%%%%%%%%%%%%%%%%
\begin{thebibliography}{99}
\label{literatura}
\addcontentsline{toc}{chapter}{Literatúra}

\bibitem{dynamips}
EasyPass Computer Training Centre. {\it Dynamips / Dynagen Tutorial}. [Online] [cit. 2018-03-25]. \\ 
Dostupné na: <\url{http://www.iteasypass.com/Dynamips.htm}>.

\bibitem{obr_dynamips_dynagen}
EasyPass Computer Training Centre. {\it Dynamips / Dynagen Tutorial}. [Online] [cit. 2018-03-25]. \\ 
Dostupné na: <http://www.iteasypass.com/Dynamips%20-%20Dynagen%20Tutorial.files/image002.gif>

\bibitem{dynamips_github}
DUPONCHELLE Julien, LINTOTT Daniel, GROSSMANN Jeremy (07-24-2017) {\it GNS3/dynamips: Dynamips development}. [Online] [cit. 2018-03-25]. \\ 
Dostupné na: <\url{https://github.com/GNS3/dynamips}>.

\bibitem{dynamips_nil}
NIL - Network Information Library (04-11-2014) {\it Connecting Dynamips/Dynagen router with a real network - in linux | NIL - Network Information Library}. [Online] [cit. 2018-03-25]. \\ 
Dostupné na: <\url{http://nil.uniza.sk/network-simulation-and-modelling/dynamipsdynagen/connecting-dynamipsdynagen-router-real-network-linux}>.

\bibitem{webiou_unetlab_unetlabv2}
Route Reflector Labs (03-21-2018) {\it Unified Networking Lab v2 (UNetLabv2) | Andrea Dainese}. [Online] [cit. 2018-03-25]. \\
Dostupné na: <\url{http://www.routereflector.com/unetlab/}>.

\bibitem{webiou_github}
DAINESE Andrea (05-30-2016) {\it dainok/iou-web}. [Online] [cit. 2018-03-25]. \\
Dostupné na: <\url{https://github.com/dainok/iou-web}>.

\bibitem{webiou_firewall_cx}
WANG Jack (2017) {\it Cisco IOU}. [Online] [cit. 2018-03-25]. \\
Dostupné na: <\url{http://www.firewall.cx/cisco-technical-knowledgebase/cisco-services-tech/1172-cisco-virl-virtual-internet-routing-lab-introduction.html}>.

\bibitem{webiou_80211}
FORDHAM Stuart (06-15-2013) {\it Getting started with Cisco IOU - IOS on Unix - Part 1 | www.802101.com}. [Online] [cit. 2018-03-25]. \\
Dostupné na: <\url{https://www.802101.com/getting-started-cisco-iou-ios-unix/}>.

\bibitem{webiou_real_network}
FERRO Greg (04-17-2011) {\it Cisco IOU:Connect IOU with real or external networks — EtherealMind}. [Online] [cit. 2018-03-25]. \\
Dostupné na: <\url{http://etherealmind.com/cisco-iou-external-real-network-remote/}>.

\bibitem{obr_webiou}
FARES Ryan (12-10-2015) {\it Cisco IOU Web Interface – netbrainstlearn}. [Online] [cit. 2018-03-25]. \\
Dostupné na: <\url{https://i2.wp.com/www.routereflector.com/images/posts/2012/09/iou-web-new.png}>.

\bibitem{virl_cisco}
Cisco Systems, Inc. (2018) {\it Cisco Virtual Internet Routing Lab Personal Edition (VIRL PE) 20 Nodes - The Cisco Learning Network}. [Online] [cit. 2018-03-26]. \\
Dostupné na: <\url{https://learningnetworkstore.cisco.com/virtual-internet-routing-lab-virl/cisco-personal-edition-pe-20-nodes-virl-20}>.

\bibitem{virl_interfacett_1}
Interface Technical Training (08-14-2015) {\it How to create a simple network topology using Cisco VIRL}. [Online] [cit. 2018-03-26]. \\
Dostupné na: <\url{https://learningnetworkstore.cisco.com/virtual-internet-routing-lab-virl/cisco-personal-edition-pe-20-nodes-virl-20}>.

\bibitem{virl_interfacett_2}
Interface Technical Training (08-28-2015) {\it How to interact with a simple network topology built using Cisco’s VIRL}. [Online] [cit. 2018-03-26]. \\
Dostupné na: <\url{https://www.interfacett.com/blogs/how-to-interact-with-a-simple-network-topology-built-using-ciscos-virl/}>.

\bibitem{virl_ciscoskills}
Cisco Skills (01-07-2017) {\it Cisco VIRL and Windows VMs}. [Online] [cit. 2018-03-26]. \\
Dostupné na: <\url{https://ciscoskills.net/2017/01/07/cisco-virl-and-windows-vms/}>.

\bibitem{virl_speaknetworks}
WANG Jack (07-27-2015) {\it Cisco VIRL External Connectivity – Speak Network Solutions}. [Online] [cit. 2018-03-26]. \\
Dostupné na: <\url{https://www.speaknetworks.com/cisco-virl-external-connectivity/}>.

\bibitem{virl_cisco_features}
Cisco Systems, Inc. (2018) {\it Virtual Internet Routing Lab (VIRL) Features - The Cisco Learning Network Store}. [Online] [cit. 2018-03-26]. \\
Dostupné na: <\url{https://learningnetworkstore.cisco.com/virlfaq/features}>.

\bibitem{virl_edition_differences}
LIU Wen, Inc. (04-28-2016) {\it VIRL Personal Edition vs. the Academic Edition - 30411 - The Cisco Learning Network}. [Online] [cit. 2018-03-26]. \\
Dostupné na: <\url{https://learningnetwork.cisco.com/docs/DOC-30411}>.

\bibitem{obr_virl_vmmaestro}
Interface Technical Training (08-28-2015) {\it How to interact with a simple network topology built using Cisco’s VIRL}. [Online] [cit. 2018-03-26]. \\
Dostupné na: <\url{https://www.interfacett.com/wp-content/uploads/2015/08/013-interact-with-simple-network-topology-in-Cisco-VIRL.jpg}>.

\bibitem{obr_virl_web}
Interface Technical Training (08-28-2015) {\it How to interact with a simple network topology built using Cisco’s VIRL}. [Online] [cit. 2018-03-26]. \\
Dostupné na: <\url{https://www.interfacett.com/wp-content/uploads/2015/08/013-interact-with-simple-network-topology-in-Cisco-VIRL.jpg}>.

\bibitem{unetlab_github}
DAINESE Andrea (05-23-2017) {\it dainok/iou-web}. [Online] [cit. 2018-03-25]. \\
Dostupné na: <\url{https://github.com/dainok/iou-web}>.

\bibitem{obr_unetlab_web}
HAGEN, LAN-Monitor.de  (03-19-2016) {\it Was ist UNetLab? – LAN-Monitor.de}. [Online] [cit. 2018-03-26]. \\
Dostupné na: <\url{https://www.lan-monitor.de/wp-content/uploads/was-ist-unetlab-01-test-lab.png}>.

\bibitem{obr_unetlabv2_arch}
Route Reflector Labs (03-21-2018) {\it Unified Networking Lab v2 (UNetLabv2) | Andrea Dainese}. [Online] [cit. 2018-03-25]. \\
Dostupné na: <\url{http://www.routereflector.com/images/unetlab/unetlab-architecture.png}>.

\end{thebibliography}

	%  Literatura

\end{document}


